\documentclass[a4paper]{jpconf}
\usepackage{graphicx}
\usepackage{wrapfig}


\begin{document}
\title{A search for $tWZ$ production in the Full Run 2 ATLAS dataset using events with four leptons.}

\author{Jake Reich}

\address{
Supervisor: Dr. James Keaveney\\
Co-Supervisor: Dr. Sahal Yacoob \\ \vspace{0.2cm}
University of Cape Town (UCT)}

\ead{jake.reich@cern.ch}

\begin{abstract}
A search for $tWZ$ production using 139 fb$^{-1}$ of $pp$ collision data at a centre-of-mass energy of $\sqrt{s} = 13$ TeV, recorded by the ATLAS experiment at CERN, is presented. Events that contain exactly four leptons (electrons or muons) are selected. Additional criteria are applied based on the number of jets, the number of $b$-tagged jets and the number of $Z$ boson candidates to define signal and control regions. The large $t\bar{t}Z$ and $ZZ$ backgrounds are distinguished from signal by a BDT-based algorithm. Inputs to the BDT-based algorithm include, a kinematic reconstruction algorithm which reconstructs leptonically decaying top quarks and a variable constructed from the output of a BDT-based algorithm which aims to classify $\ell b$ systems originating from top quarks. The expected signal strength is extracted via a blinded maximum-likelihood fit to multiple signal and control regions. The expected signal strength was measured as, $\mu (tWZ) = 1.91^{+0.95}_{-0.82}$, leading to an expected significance of 1.44$\sigma$. An expected upper limit on the signal strength is set to $\mu_{up}^{exp} =   1.61^{+2.35}_{-1.16}$. Furthermore, a combined blinded maximum-likelihood fit was performed across the tetralepton channel (four lepton final state) and an independent analysis of the trilepton channel (three lepton final state), to further increase the sensitivity this analysis to the $tWZ$ cross section. The expected signal strength was measured as, $\mu (tWZ) = 1.80^{+0.70}_{-0.65}$, leading to an expected significance of 1.61$\sigma$. An expected upper limit on the signal strength is set to $\mu_{up}^{exp} =   1.43^{+2.04}_{-1.03}$.
\end{abstract}

\section{Introduction}
The production of a single top quark in association with a $W^{\pm}$ and $Z$ boson ($tWZ$) at the CERN LHC is sensitive to both the neutral and charged electroweak couplings of the top quark as the process involves the simultaneous production of a $W$ boson and a $Z$ boson in association with the top quark. Due to the very large coupling of the top quark to the Higgs boson, the electroweak couplings of the top quark are a theoretically well-motivated area in which to search for the first signs of new physics~\cite{TopQuarkCouplings-searchNewPhysics, top-a-tool-for-disc} that could offer a resolution to the Hierarchy Problem~\cite{hierarchy-problem-paper, Burdman_2015}. The recent lack of signs of new physics from LHC data tells us that new physics is either very heavy, or is very weakly coupled to Standard Model (SM) particles, therefore signs of new physics might only be observed in anomalous rates of well-chosen processes. A prime example of such a process is $tWZ$. This has an extremely low production cross section ($\approx 160$ fb for $\sqrt{s}=13$ TeV~\cite{twz-theory-paper}), meaning that it is an extremely rare process and subsequently, it has never been observed by any particle physics experiment.\\

\noindent
The latest datasets recorded by the ATLAS experiment at the CERN LHC are sufficiently large to potentially allow for an observation of $tWZ$ production. In this analysis, the Full Run 2 dataset recorded by ATLAS is used to search for $tWZ$ production in the tetralepton channel ($tWZ$ with exactly four final state leptons). A kinematic reconstruction technique is used which aims to discriminate between $tWZ$ and our most prominent background process, $t\bar{t}Z$. In addition to this, Machine Learning techniques are implemented to further isolate our $tWZ$ signal. Backgrounds from Standard Model processes including those in which one or more leptons originate from the semileptonic decay of a heavy hadron or a photon conversion are estimated by fitting predictions from simulation to data in dedicated control regions. As this work forms the basis of an official ATLAS analysis, only blinded results are shown. A maximum likelihood fit is performed over our two $tWZ$ signal regions and three control regions, to measure the cross section of $tWZ$ in the tetralepton channel. In this thesis only the tetralepton channel is explored, however, the results of a combined fit that includes an entirely independent analysis of the trilepton channel is presented with no overlap of events between the regions defined for these channels. 


\section{$tWZ$}
The main backgrounds for $tWZ$ in the tetralepton channel are the production of a two tops, both in the $\ell \nu b$~\footnote{In this document, $\ell$ refers to an electron or muon, $\nu$ refers to a neutrino or anti-neutrino and $b$ refers to a bottom quark or anti-bottom quark} final state channel, together with a $Z$ boson ($t\bar{t}Z$) and diboson production with fully leptonic final states ($ZZ$). In Figure \ref{fig:twz-feynamn}, the Leading Order (LO) Feynman diagram for $tWZ$ in the tetralepton channel and it's major background processes, $t\bar{t}Z$ and $ZZ$ are shown.

\begin{figure}[h]
\includegraphics[width=12pc]{4lep_feynman.png}
\includegraphics[width=12pc]{ttz_feynman.png}
\includegraphics[width=12pc]{zz_feynman.png}
\caption{The LO Feynman diagrams of $tWZ$ production in the tetralepton channel (left) and it's major backgrounds, $t\bar{t}Z$ (middle) and $ZZ$ (right) are shown.}
\label{fig:twz-feynamn}
\end{figure}
\noindent
Prior to this analysis, only three experimental studies of $tWZ$ in ATLAS have been performed. Two of the studies utilised the trilepton channel to search for $tWZ$ production, whereas the third study utilised both the tri- and tetralepton channels. The first search utilised 36 fb$^{-1}$ of ATLAS data and an upper limit on the cross section of $tWZ$ was set at a value of $\approx$ 6 times the SM cross section~\cite{twz_3_lep}. The second search in the trilepton channel utilised 139 $^{-1}$ of ATLAS data and an expected upper limit on the cross section of $tWZ$ was set at a value of $\approx$ 2.6 times the SM cross section~\cite{ben-thesis}. In Section \ref{sec:combined-results}, the aformentioned analysis will be used in combination with this analysis, in order to further increase the sensitivity of the cross section of $tWZ$. The third study investigated the feasibility of a cross section measurement of $tWZ$ production with CMS Run 3 data (300 fb$^{-1}$)~\cite{Tschida:2020ftz}, by utilising the trilepton and tetralepton channels. The study showed that it is possible to exclude $\mu (tWZ)$ at the $7\sigma$ significance level using 300 fb$^{-1}$ of data. This study needs to be further investigated, since its findings seem improbable given the results obtained in this thesis. 
\begin{wrapfigure}{r}{0.4\textwidth}
  \begin{center}
    \includegraphics[width=0.4\textwidth]{3lep_feynman.png}
  \end{center}
  \caption{The LO Feynman diagrams of $tWZ$ production in the trilepton channel is shown.}
\end{wrapfigure}
The most apparent difference between the trilepton and tetralepton channels is their branching ratios ($\frac{\Gamma_i}{\Gamma}$), with the tetralepton channel's branching ratio being much smaller than that of the trilepton channel. The tetralepton channel has a cross section times branching ratio of $\sigma^{\text{NLO}}_{(tW^{\pm}Z).Br(4\ell)} = 0.7$ fb~\cite{twz-theory-paper}, and is therefore an extremely rare process. The trilepton channel has a cross section times branching ratio of $\sigma^{\text{NLO}}_{(tW^{\pm}Z).Br(3\ell)} = 3.9$ fb~\cite{twz-theory-paper}, which is around a factor of four larger than that of the tetralepton channel. Despite the tetralepton channel's low statistics, it is not subject to the large $WZ$ background present in the trilepton channel~\cite{ben-thesis}. The tetralepton channel has a substantial amount of $ZZ$ background (not present in the trilepton channel), fortunately this can be easily suppressed due to the full reconstructability of the two leptonically decaying $Z$-bosons.

\section{The search for $tWZ$ production in the Full Run 2 ATLAS dataset using events with four leptons}

\subsection{Regions and Event Selection}
In order to suppress potential fakes and quarkonia (low mass resonances such as $J/\psi$ and upsilon) a requirement that all OSSF lepton pairs have an invariant mass, $m_{\text{OSSF}}$, greater than $\SI{10}{\GeV}$ is applied. The final state lepton charges must sum to zero. Therefore a requirement of $\sum_{i=1}^{4} \text{charge}(\ell_i) = 0$ is applied. The invariant mass of the OSSF lepton pair coming from the $Z$ boson must equal the invariant mass of the $Z$ boson, and noting that $e$,$\mu$ reconstruction and identification in the ATLAS detector has a high efficiency~\cite{Marchese:2688452}, these OSSF leptons are used to reconstruct $Z$ bosons with relatively high confidence. A $Z$ candidate is defined in this analysis as an OSSF lepton pair with an invariant mass, $m_{\text{OSSF}}$, satisfying the condition, $|m_{\text{OSSF}} - m_Z| <  \SI{30}{\GeV}$, where $m(Z)$ is the nominal $Z$ boson mass ($91.1876$ GeV~\cite{pdg}). This wider mass window is used in order to cover the full range of the $m(Z)$ distribution, in an attempt to increase the number of events which pass our baseline selections. Multiple $Z$ candidates can be present in certain decay channels (e.g. $eeee$, $\mu\mu ee$, $\mu \mu \mu \mu$). In these cases, the $Z$ candidate which has an invariant mass closest to the nominal $Z$ boson mass is chosen.\\ 

Two $tWZ$ SRs are defined in an attempt to suppress and constrain the $ZZ$ background. The two $tWZ$ SR's differ by the flavours of their leptons which don't originate from the decay of a $Z$-boson (non-$Z$ leptons). The $ZZ$ background has two $Z$-bosons which decay into a pair of OSSF lepton pairs, in order to mimic the $tWZ$ signal. This is taken advantage of, to define a $tWZ$ region enrich in $ZZ$ background and one with a minimal $ZZ$ background component. This is done by requiring that one of the $tWZ$ SRs has its two non-$Z$ leptons to have opposite flavour and the other $tWZ$ SR is required to have its non-$Z$ leptons to have the same flavour. These two disjoint $tWZ$ SRs are named $tWZ$ OF SR and $tWZ$ SF SR respectively. It is therefore expected that the $tWZ$ SF SR contains the majority of the $ZZ$ background events across both $tWZ$ SRs. In order to check the modelling of the most dominant background components in our signal region, $t\bar{t}Z$ and $ZZb$ control regions are defined. The $t\bar{t}Z$ control region has the same requirement on the number of reconstructed $Z$ boson candidates in the signal region (due to a commonality on the number of $Z$ bosons present in both processes), however it is required that there are at least two jets and that exactly two of these jets are $b$-tagged (corresponding to the $b$-quark jets originating from the two top quark decays). A $ZZb$ region is defined, as opposed to a $ZZ$ region, since the $ZZ$ background present in the $tWZ$ signal region contains exactly one b-tagged jet. Therefore defining a region with $ZZ$ plus exactly one $b$-jet more closely resembles the $ZZ$ background present in the signal region. In addition to this, mis-modelling of $ZZ$ has been seen in other analyses~\cite{Aaboud:2019, ppToZZ:CMSpaper}, further motivating the use of a $ZZb$ control region over a $ZZ$ CR. The $ZZb$ CR requires exactly two $Z$ boson candidates and exactly one b-tagged jet, resulting in an implicit requirement on the number of jets ($N_{jet} \geq 1$). In order to constrain the fake lepton component contained within the $t\bar{t}Z$ sample, a $(tWZ)_{\text{fake}}$ CR is defined which is as similar as possible to the $tWZ$ SRs but is enhanced in fakes. This is achieved by defining the $(tWZ)_{\text{fake}}$ CR to inherit the same selection criteria as the $tWZ$ SRs however, in this case, a requirement of exactly 3 tight leptons and exactly 1 loose (and NOT tight) lepton is applied. Loose leptons are required in this region, since looser leptons are more likely to be fakes compared to tighter leptons. In Table \ref{tab:4Lep-cutsummary}, a summary of the final selection criteria and region definitions is shown.

\begin{table}[h!]
\centering
\resizebox{\textwidth}{!}{%
\begin{tabular}{ccccc} 
\hline
\multicolumn{5}{c}{\textbf{Baseline selections}}   \\ 
\hline
\multicolumn{5}{c}{N$_{\ell} = 4$} \\
\multicolumn{5}{c}{$p_{T}(\ell_1,\ell_2,\ell_3,\ell_4) > (28,18,10,10)$ $\SI{}{\GeV}$} \\
\multicolumn{5}{c}{$p_{T}(\text{jet})> \SI{20}{\GeV}$, $|\eta(\text{jet})| < \SI{4.5}{}$, $\text{jvt} > \SI{0.5}{}$}   \\
\multicolumn{5}{c}{$|\eta(\ell_e)| < \SI{2.47}{}$ excluding $\SI{1.37}{} < |\eta(\ell_e)| < \SI{1.52}{}$}  \\
\multicolumn{5}{c}{$|\eta(\ell_\mu)| < \SI{2.5}{}$}\\
\multicolumn{5}{c}{$\displaystyle\sum_{i=1}^{4} \text{charge}(\ell_i) = 0$}\\
\multicolumn{5}{c}{All OSSF lepton pairs require $m_{\text{OSSF}} > \SI{10}{\GeV}$}\\ 
\hline
\multicolumn{5}{c}{\textbf{Regions}}   \\ 
\hline
$tWZ$ OF SR  & $tWZ$ SF SR & $t\bar{t}Z$ CR   & $ZZb$ CR & $(tWZ)_{\text{fake}}$ CR  \\ 
\hline
\multirow{2}{*}{N$_{\ell}$(tight) $=$ 4} & \multirow{2}{*}{N$_{\ell}$(tight) $=$ 4} & \multirow{2}{*}{N$_{\ell}$(tight) $=$ 4} & \multirow{2}{*}{N$_{\ell}$(tight) $=$ 4} & N$_{\ell}$(tight) $=$ 3   \\
 &   &  &  & N$_{\ell}$(loose and NOT tight) $=$ 1 \\
 &   &  &  &   \\
N$_{\text{Z candidate}} = 1$ & N$_{\text{Z candidate}} = 1$  & N$_{\text{Z candidate}} = 1$ & N$_{\text{Z candidate}} = 2$ & N$_{\text{Z candidate}} = 1$  \\
 &   &  &  &   \\
N$_{\text{jet}} \geq 1$  &  N$_{\text{jet}} \geq 1$  & N$_{\text{jet}} \geq 2$  & N$_{\text{jet}} \geq 1$  & N$_{\text{jet}} \geq 1$   \\
 &   &  &  &   \\
\multirow{3}{*}{N$_{\text{b-jet}}$(tight) $=$ 1} & \multirow{3}{*}{N$_{\text{b-jet}}$(tight) $=$ 1} & N$_{\text{b-jet}}$(tight) $\geq$ 1   & \multirow{3}{*}{N$_{\text{b-jet}}$(tight) $=$ 1} & \multirow{3}{*}{N$_{\text{b-jet}}$(tight) $=$ 1}  \\
 &   & N$_{\text{b-jet}}$(loose) $\geq$ 0   &  &   \\
 &   & N$_{\text{b-jet}}$(tight) $+$ N$_{\text{b-jet}}$(loose) $=2$ &  &   \\
 & & & &\\
Opp. Flavour non-Z leptons & Same Flavour non-Z leptons  & -  & - & -  \\
\hline
\end{tabular}}
	\caption{
		A summary of the requirements applied for selecting events in the signal and control regions is shown.
	}%
	\label{tab:4Lep-cutsummary}
\end{table}

\subsection{Improving signal vs background discrimination}

The presence of different numbers of top quarks is a key discriminator between signal and the dominant background process, $t\bar{t}Z$. This information is aimed to be exploited by reconstructing $\ell b$ systems as a proxy for top quarks (since, $t\rightarrow W(\rightarrow \ell \nu) b$). This is done in two ways, firstly, by implementation of a kinematic reconstruction algorithm (Two Neutrino Scanning Method) which aims to determine the likelihood of an event containing two top quarks and secondly, by implementing a Boosted Decision Tree (BDT) which is used to distinguish between $\ell b$ systems that originate from top quarks and $\ell b$ systems which do not originate from top quarks. In this analysis, this BDT is referred to as an \textit{object-level} BDT. Certain variables constructed from event information show discrimination between signal and background events. This information can be exploited to discriminate between signal and background events by constructing a BDT which uses these discriminating variables for training. A BDT is implemented and is used to discriminate between \tWZ events and its major backgrounds, \ttZ and \ZZ. Furthermore, this BDT takes information from the kinematic reconstruction algorithm and the object-level BDT in order to maximize its discriminating power. In this thesis, this BDT is referred to as an \textit{event-level} BDT. The discriminator output from the object-level BDT can be converted to a variable which can then be used as an input to the event-level BDT.

\subsubsection{Two Neutrino Scanning Method (2$\nu $SM)}
The difference in the number of resonant top quarks in the $tWZ$ signal and the dominant background, $t\bar{t}Z$, is a key feature which can be exploited in order to discriminate between these two processes. In Section \ref{sec:object-level-bdt}, a BDT was implemented which exploits this information by aiming to identify $\ell b$ systems originating from top quarks. In this section, a kinematic reconstruction algorithm (Two Neutrino Scanning Method) is implemented which exploits the same feature. The Two Neutrino Scanning Method (2$\nu$SM) algorithm\footnote{software tool and weights provided by Thomas McCarthy (\ttZ analysis group - Max Planck Institute)} aims to reconstruct $t\bar{t}$ systems in the 2$\ell$, 3$\ell$ and 4$\ell$ final states (e.g. 2$\ell$ case: $t\bar{t}\rightarrow \ell^{+}\nu_{\ell}b\ell^{-}\bar{\nu}_{\ell}\bar{b}$). The 2$\nu$SM algorithm aims to reconstruct a $t\bar{t}$ system by finding two neutrinos ($\nu_{1}$ and $\nu_{2}$) which are most likely to correspond to the neutrinos that originate from the decay of a $t\bar{t}$ system. This algorithm can be used in our analysis to discriminate between \tWZ and \ttZ, since the OSSF leptons which decay from the $Z$ boson can be easily reconstructed and removed before inputting the event into the algorithm. The removal of the $Z$ boson results in \tWZ events that don't resemble $t\bar{t}$ systems and \ttZ events that do resemble $t\bar{t}$ systems, which the algorithm is designed to distinguish between. It would then be expected that the 2$\nu$SM algorithm returns a higher score from a \ttZ event ($\sim$ 1, i.e. it resembles a $t\bar{t}$ event after removal of the $Z$ boson) and a lower score from a \tWZ event ($\sim$ 0, i.e. it does not resemble a $t\bar{t}$ event after removal of the $Z$ boson). 

\section{Preparing your paper}
\verb"jpconf" requires \LaTeXe\ and  can be used with other package files such
as those loading the AMS extension fonts 
\verb"msam" and \verb"msbm" (these fonts provide the 
blackboard bold alphabet and various extra maths symbols as well as 
symbols useful in figure captions); an extra style file \verb"iopams.sty" is 
provided to load these packages and provide extra definitions for bold Greek letters. 
\subsection{Headers, footers and page numbers}
Authors should {\it not} add headers, footers or page numbers to the pages of their article---they will
be added by \iopp\ as part of the production process.

\subsection{{\cls\ }package options}
The \cls\ class file has two options `a4paper' and `letterpaper':
\begin{verbatim}
\documentclass[a4paper]{jpconf}
\end{verbatim}

or \begin{verbatim}
\documentclass[letterpaper]{jpconf}
\end{verbatim}

\begin{center}
\begin{table}[h]
\caption{\label{opt}\cls\ class file options.}
%\footnotesize\rm
\centering
\begin{tabular}{@{}*{7}{l}}
\br
Option&Description\\
\mr
\verb"a4paper"&Set the paper size and margins for A4 paper.\\
\verb"letterpaper"&Set the paper size and margins for US letter paper.\\
\br
\end{tabular}
\end{table}
\end{center}

The default paper size is A4 (i.e., the default option is {\tt a4paper}) but this can be changed to Letter by 
using \verb"\documentclass[letterpaper]{jpconf}". It is essential that you do not put macros into the text which alter the page dimensions.

\section{The title, authors, addresses and abstract} 
The code for setting the title page information is slightly different from
the normal default in \LaTeX\ but please follow these instructions as carefully as possible so all articles within a conference have the same style to the title page. 
The title is set in bold unjustified type using the command
\verb"\title{#1}", where \verb"#1" is the title of the article. The
first letter of the title should be capitalized with the rest in lower case. 
The next information required is the list of all authors' names followed by 
the affiliations. For the authors' names type \verb"\author{#1}", 
where \verb"#1" is the 
list of all authors' names. The style for the names is initials then
surname, with a comma after all but the last 
two names, which are separated by `and'. Initials should {\it not} have 
full stops. First names may be used if desired. The command \verb"\maketitle" is not
required.

The addresses of the authors' affiliations follow the list of authors. 
Each address should be set by using
\verb"\address{#1}" with the address as the single parameter in braces. 
If there is more 
than one address then a superscripted number, followed by a space, should come at the start of
each address. In this case each author should also have a superscripted number or numbers following their name to indicate which address is the appropriate one for them.
 
Please also provide e-mail addresses for any or all of the authors using an \verb"\ead{#1}" command after the last address. \verb"\ead{#1}" provides the text Email: so \verb"#1" is just the e-mail address or a list of emails.  

The abstract follows the addresses and
should give readers concise information about the content 
of the article and should not normally exceed 200 
words. {\bf All articles must include an abstract}. To indicate the start 
of the abstract type \verb"\begin{abstract}" followed by the text of the 
abstract.  The abstract should normally be restricted 
to a single paragraph and is terminated by the command
\verb"\end{abstract}"

\subsection{Sample coding for the start of an article}
\label{startsample}
The code for the start of a title page of a typical paper might read:
\begin{verbatim}
\title{The anomalous magnetic moment of the 
neutrino and its relation to the solar neutrino problem}

\author{P J Smith$^1$, T M Collins$^2$, 
R J Jones$^{3,}$\footnote[4]{Present address:
Department of Physics, University of Bristol, Tyndalls Park Road, 
Bristol BS8 1TS, UK.} and Janet Williams$^3$}

\address{$^1$ Mathematics Faculty, Open University, 
Milton Keynes MK7~6AA, UK}
\address{$^2$ Department of Mathematics, 
Imperial College, Prince Consort Road, London SW7~2BZ, UK}
\address{$^3$ Department of Computer Science, 
University College London, Gower Street, London WC1E~6BT, UK}

\ead{williams@ucl.ac.uk}

\begin{abstract}
The abstract appears here.
\end{abstract}
\end{verbatim}

\section{The text}
The text of the article should should be produced using standard \LaTeX\ formatting. Articles may be divided into sections and subsections, but the length limit provided by the \corg\ should be adhered to.

\subsection{Acknowledgments}
Authors wishing to acknowledge assistance or encouragement from 
colleagues, special work by technical staff or financial support from 
organizations should do so in an unnumbered Acknowledgments section 
immediately following the last numbered section of the paper. The 
command \verb"\ack" sets the acknowledgments heading as an unnumbered
section.

\subsection{Appendices}
Technical detail that it is necessary to include, but that interrupts 
the flow of the article, may be consigned to an appendix. 
Any appendices should be included at the end of the main text of the paper, after the acknowledgments section (if any) but before the reference list.
If there are two or more appendices they will be called Appendix A, Appendix B, etc. 
Numbered equations will be in the form (A.1), (A.2), etc,
figures will appear as figure A1, figure B1, etc and tables as table A1,
table B1, etc.

The command \verb"\appendix" is used to signify the start of the
appendixes. Thereafter \verb"\section", \verb"\subsection", etc, will 
give headings appropriate for an appendix. To obtain a simple heading of 
`Appendix' use the code \verb"\section*{Appendix}". If it contains
numbered equations, figures or tables the command \verb"\appendix" should
precede it and \verb"\setcounter{section}{1}" must follow it. 

\section{References}
%%%%%%%%%%%%%%%%%%%%%%%%%%%%%%%%%%%%%%%%%%%
In the online version of \jpcs\ references will be linked to their original source or to the article within a secondary service such as INSPEC or ChemPort wherever possible. To facilitate this linking extra care should be taken when preparing reference lists. 

Two different styles of referencing are in common use: the Harvard alphabetical system and the Vancouver numerical system.  For \jpcs, the Vancouver numerical system is preferred but authors should use the Harvard alphabetical system if they wish to do so. In the numerical system references are numbered sequentially throughout the text within square brackets, like this [2], and one number can be used to designate several references.  

\subsection{Using \BibTeX}
We highly recommend the {\ttfamily\textbf\selectfont iopart-num} \BibTeX\ package by Mark~A~Caprio \cite{iopartnum}, which is included with this documentation.

\subsection{Reference lists}
A complete reference should provide the reader with enough information to locate the article concerned, whether published in print or electronic form, and should, depending on the type of reference, consist of:  

\begin{itemize}
\item name(s) and initials;
\item date published;
\item title of journal, book or other publication; 
\item titles of journal articles may also be included (optional);
\item volume number;
\item editors, if any;
\item town of publication and publisher in parentheses for {\it books};
\item the page numbers.
\end{itemize}

Up to ten authors may be given in a particular reference; where 
there are more than ten only the first should be given followed by 
`{\it et al}'. If an author is unsure of a particular journal's abbreviated title it is best to leave the title in 
full. The terms {\it loc.\ cit.\ }and {\it ibid.\ }should not be used. 
Unpublished conferences and reports should generally not be included 
in the reference list and articles in the course of publication should 
be entered only if the journal of publication is known. 
A thesis submitted for a higher degree may be included 
in the reference list if it has not been superseded by a published 
paper and is available through a library; sufficient information 
should be given for it to be traced readily.

\subsection{Formatting reference lists}
Numeric reference lists should contain the references within an unnumbered section (such as \verb"\section*{References}"). The 
reference list itself is started by the code 
\verb"\begin{thebibliography}{<num>}", where \verb"<num>" is the largest
number in the reference list and is completed by
\verb"\end{thebibliography}". 
Each reference starts with \verb"\bibitem{<label>}", where `label' is the label used for cross-referencing. Each \verb"\bibitem" should only contain a reference to a single article (or a single article and a preprint reference to the same article).  When one number actually covers a group of two or more references to different articles, \verb"\nonum"
should replace \verb"\bibitem{<label>}" at
the start of each reference in the group after the first.

For an alphabetic reference list use \verb"\begin{thereferences}" ... \verb"\end{thereferences}" instead of the
`thebibliography' environment and each reference can be start with just \verb"\item" instead of \verb"\bibitem{label}"
as cross referencing is less useful for alphabetic references.

\subsection {References to printed journal articles}
A normal reference to a journal article contains three changes of font (see table \ref{jfonts}) and is constructed as follows:

\begin{itemize}
\item the authors should be in the form surname (with only the first letter capitalized) followed by the initials with no periods after the initials. Authors should be separated by a comma except for the last two which should be separated by `and' with no comma preceding it;
\item the article title (if given) should be in lower case letters, except for an initial capital, and should follow the date;
\item the journal title is in italic and is abbreviated. If a journal has several parts denoted by different letters the part letter should be inserted after the journal in Roman type, e.g. {\it Phys. Rev.} A;
\item the volume number should be in bold type;
\item both the initial and final page numbers should be given where possible. The final page number should be in the shortest possible form and separated from the initial page number by an en rule `-- ', e.g. 1203--14, i.e. the numbers `12' are not repeated.
\end{itemize}

A typical (numerical) reference list might begin

\medskip
\begin{thebibliography}{9}
\item Strite S and Morkoc H 1992 {\it J. Vac. Sci. Technol.} B {\bf 10} 1237 
\item Jain S C, Willander M, Narayan J and van Overstraeten R 2000 
{\it J. Appl. Phys}. {\bf 87} 965 
\item Nakamura S, Senoh M, Nagahama S, Iwase N, Yamada T, Matsushita T, Kiyoku H 
and 	Sugimoto Y 1996 {\it Japan. J. Appl. Phys.} {\bf 35} L74 
\item Akasaki I, Sota S, Sakai H, Tanaka T, Koike M and Amano H 1996 
{\it Electron. Lett.} {\bf 32} 1105 
\item O'Leary S K, Foutz B E, Shur M S, Bhapkar U V and Eastman L F 1998 
{\it J. Appl. Phys.} {\bf 83} 826 
\item Jenkins D W and Dow J D 1989 {\it Phys. Rev.} B {\bf 39} 3317 
\end{thebibliography}
\smallskip

\noindent which would be obtained by typing

\begin{verbatim}
\begin{\thebibliography}{9}
\item Strite S and Morkoc H 1992 {\it J. Vac. Sci. Technol.} B {\bf 10} 1237 
\item Jain S C, Willander M, Narayan J and van Overstraeten R 2000 
{\it J. Appl. Phys}. {\bf 87} 965 
\item Nakamura S, Senoh M, Nagahama S, Iwase N, Yamada T, Matsushita T, Kiyoku H 
and 	Sugimoto Y 1996 {\it Japan. J. Appl. Phys.} {\bf 35} L74 
\item Akasaki I, Sota S, Sakai H, Tanaka T, Koike M and Amano H 1996 
{\it Electron. Lett.} {\bf 32} 1105 
\item O'Leary S K, Foutz B E, Shur M S, Bhapkar U V and Eastman L F 1998 
{\it J. Appl. Phys.} {\bf 83} 826 
\item Jenkins D W and Dow J D 1989 {\it Phys. Rev.} B {\bf 39} 3317 
\end{\thebibliography}
\end{verbatim}

\begin{center}
\begin{table}[h]
\centering
\caption{\label{jfonts}Font styles for a reference to a journal article.} 
\begin{tabular}{@{}l*{15}{l}}
\br
Element&Style\\
\mr
Authors&Roman type\\
Date&Roman type\\
Article title (optional)&Roman type\\
Journal title&Italic type\\
Volume number&Bold type\\
Page numbers&Roman type\\
\br
\end{tabular}
\end{table}
\end{center}

\subsection{References to \jpcs\ articles}
Each conference proceeding published in \jpcs\ will be a separate {\it volume}; 
references should follow the style for conventional printed journals. For example:\vspace{6pt}
\numrefs{1}
\item Douglas G 2004 \textit{J. Phys.: Conf. Series} \textbf{1} 23--36
\endnumrefs

%%%%%%%%%%%%%%%%%%%%%%%%%%%%%%%%%%
\subsection{References to preprints}
For preprints there are two distinct cases:
\renewcommand{\theenumi}{\arabic{enumi}}
\begin{enumerate}
\item Where the article has been published in a journal and the preprint is supplementary reference information. In this case it should be presented as:
\medskip
\numrefs{1}
\item Kunze K 2003 T-duality and Penrose limits of spatially homogeneous and inhomogeneous cosmologies {\it Phys. Rev.} D {\bf 68} 063517 ({\it Preprint} gr-qc/0303038)
\endnumrefs
\item Where the only reference available is the preprint. In this case it should be presented as
\medskip
\numrefs{1}
\item Milson R, Coley A, Pravda V and Pravdova A 2004 Alignment and algebraically special tensors {\it Preprint} gr-qc/0401010
\endnumrefs
\end{enumerate}

\subsection{References to electronic-only journals}
In general article numbers are given, and no page ranges, as most electronic-only journals start each article on page 1.

\begin{itemize} 
\item For {\it New Journal of Physics} (article number may have from one to three digits)
\numrefs{1}
\item Fischer R 2004 Bayesian group analysis of plasma-enhanced chemical vapour deposition data {\it New. J. Phys.} {\bf 6} 25 
\endnumrefs
\item For SISSA journals the volume is divided into monthly issues and these form part of the article number

\numrefs{2}
\item Horowitz G T and Maldacena J 2004 The black hole final state {\it J. High Energy Phys.}  	JHEP02(2004)008
\item Bentivegna E, Bonanno A and Reuter M 2004 Confronting the IR fixed point cosmology 	with 	high-redshift observations {\it J. Cosmol. Astropart. Phys.} JCAP01(2004)001  
\endnumrefs
\end{itemize} 

\subsection{References to books, conference proceedings and reports}
References to books, proceedings and reports are similar to journal references, but have 
only two changes of font (see table~\ref{book}). 

\begin{table}
\centering
\caption{\label{book}Font styles for references to books, conference proceedings and reports.}
\begin{tabular}{@{}l*{15}{l}}
\br
Element&Style\\
\mr
Authors&Roman type\\
Date&Roman type\\
Book title (optional)&Italic type\\
Editors&Roman type\\
Place (city, town etc) of publication&Roman type\\
Publisher&Roman type\\
Volume&Roman type\\
Page numbers&Roman type\\
\br
\end{tabular}
\end{table}

Points to note are:
\medskip
\begin{itemize}
\item Book titles are in italic and should be spelt out in full with initial capital letters for all except minor words. Words such as Proceedings, Symposium, International, Conference, Second, etc should be abbreviated to {\it Proc.}, {\it Symp.}, {\it Int.}, {\it Conf.}, {\it 2nd}, respectively, but the rest of the title should be given in full, followed by the date of the conference and the town or city where the conference was held. For Laboratory Reports the Laboratory should be spelt out wherever possible, e.g. {\it Argonne National Laboratory Report}.
\item The volume number, for example vol 2, should be followed by the editors, if any, in a form such as `ed A J Smith and P R Jones'. Use {\it et al} if there are more than two editors. Next comes the town of publication and publisher, within brackets and separated by a colon, and finally the page numbers preceded by p if only one number is given or pp if both the initial and final numbers are given.
\end{itemize}

Examples taken from published papers:
\medskip

\numrefs{99}
\item Kurata M 1982 {\it Numerical Analysis for Semiconductor Devices} (Lexington, MA: Heath)
\item Selberherr S 1984 {\it Analysis and Simulation of Semiconductor Devices} (Berlin: Springer)
\item Sze S M 1969 {\it Physics of Semiconductor Devices} (New York: Wiley-Interscience)
\item Dorman L I 1975 {\it Variations of Galactic Cosmic Rays} (Moscow: Moscow State University Press) p 103
\item Caplar R and Kulisic P 1973 {\it Proc. Int. Conf. on Nuclear Physics (Munich)} vol 1 (Amsterdam: 	North-Holland/American Elsevier) p 517
\item Cheng G X 2001 {\it Raman and Brillouin Scattering-Principles and Applications} (Beijing: Scientific) 
\item Szytula A and Leciejewicz J 1989 {\it Handbook on the Physics and Chemistry of Rare Earths} vol 12, ed K A Gschneidner Jr and L Erwin (Amsterdam: Elsevier) p 133
\item Kuhn T 1998 {\it Density matrix theory of coherent ultrafast dynamics Theory of Transport Properties of Semiconductor Nanostructures} (Electronic Materials vol 4) ed E Sch\"oll (London: Chapman and Hall) chapter 6 pp 173--214
\endnumrefs

\section{Tables and table captions}
Tables should be numbered serially and referred to in the text 
by number (table 1, etc, {\bf rather than} tab. 1). Each table should be a float and be positioned within the text at the most convenient place near to where it is first mentioned in the text. It should have an 
explanatory caption which should be as concise as possible. 

\subsection{The basic table format}
The standard form for a table is:
\begin{verbatim}
\begin{table}
\caption{\label{label}Table caption.}
\begin{center}
\begin{tabular}{llll}
\br
Head 1&Head 2&Head 3&Head 4\\
\mr
1.1&1.2&1.3&1.4\\
2.1&2.2&2.3&2.4\\
\br
\end{tabular}
\end{center}
\end{table}
\end{verbatim}

The above code produces table~\ref{ex}.

\begin{table}[h]
\caption{\label{ex}Table caption.}
\begin{center}
\begin{tabular}{llll}
\br
Head 1&Head 2&Head 3&Head 4\\
\mr
1.1&1.2&1.3&1.4\\
2.1&2.2&2.3&2.4\\
\br
\end{tabular}
\end{center}
\end{table}

Points to note are:
\medskip
\begin{enumerate}
\item The caption comes before the table.
\item The normal style is for tables to be centred in the same way as
equations. This is accomplished
by using \verb"\begin{center}" \dots\ \verb"\end{center}".

\item The default alignment of columns should be aligned left.

\item Tables should have only horizontal rules and no vertical ones. The rules at
the top and bottom are thicker than internal rules and are set with
\verb"\br" (bold rule). 
The rule separating the headings from the entries is set with
\verb"\mr" (medium rule). These commands do not need a following double backslash.

\item Numbers in columns should be aligned as appropriate, usually on the decimal point;
to help do this a control sequence \verb"\lineup" has been defined 
which sets \verb"\0" equal to a space the size of a digit, \verb"\m"
to be a space the width of a minus sign, and \verb"\-" to be a left
overlapping minus sign. \verb"\-" is for use in text mode while the other
two commands may be used in maths or text.
(\verb"\lineup" should only be used within a table
environment after the caption so that \verb"\-" has its normal meaning
elsewhere.) See table~\ref{tabone} for an example of a table where
\verb"\lineup" has been used.
\end{enumerate}

\begin{table}[h]
\caption{\label{tabone}A simple example produced using the standard table commands 
and $\backslash${\tt lineup} to assist in aligning columns on the 
decimal point. The width of the 
table and rules is set automatically by the 
preamble.} 

\begin{center}
\lineup
\begin{tabular}{*{7}{l}}
\br                              
$\0\0A$&$B$&$C$&\m$D$&\m$E$&$F$&$\0G$\cr 
\mr
\0\023.5&60  &0.53&$-20.2$&$-0.22$ &\01.7&\014.5\cr
\0\039.7&\-60&0.74&$-51.9$&$-0.208$&47.2 &146\cr 
\0123.7 &\00 &0.75&$-57.2$&\m---   &---  &---\cr 
3241.56 &60  &0.60&$-48.1$&$-0.29$ &41   &\015\cr 
\br
\end{tabular}
\end{center}
\end{table}
 
\section{Figures and figure captions}
Figures must be included in the source code of an article at the appropriate place in the text not grouped together at the end. 

Each figure should have a brief caption describing it and, if 
necessary, interpreting the various lines and symbols on the figure. 
As much lettering as possible should be removed from the figure itself and 
included in the caption. If a figure has parts, these should be 
labelled ($a$), ($b$), ($c$), etc. 
\Tref{blobs} gives the definitions for describing symbols and lines often
used within figure captions (more symbols are available
when using the optional packages loading the AMS extension fonts).

\begin{table}[h]
\caption{\label{blobs}Control sequences to describe lines and symbols in figure 
captions.}
\begin{center}
\begin{tabular}{lllll}
\br
Control sequence&Output&&Control sequence&Output\\
\mr
\verb"\dotted"&\dotted        &&\verb"\opencircle"&\opencircle\\
\verb"\dashed"&\dashed        &&\verb"\opentriangle"&\opentriangle\\
\verb"\broken"&\broken&&\verb"\opentriangledown"&\opentriangledown\\
\verb"\longbroken"&\longbroken&&\verb"\fullsquare"&\fullsquare\\
\verb"\chain"&\chain          &&\verb"\opensquare"&\opensquare\\
\verb"\dashddot"&\dashddot    &&\verb"\fullcircle"&\fullcircle\\
\verb"\full"&\full            &&\verb"\opendiamond"&\opendiamond\\
\br
\end{tabular}
\end{center}
\end{table}


Authors should try and use the space allocated to them as economically as possible. At times it may be convenient to put two figures side by side or the caption at the side of a figure. To put figures side by side, within a figure environment, put each figure and its caption into a minipage with an appropriate width (e.g. 3in or 18pc if the figures are of equal size) and then separate the figures slightly by adding some horizontal space between the two minipages (e.g. \verb"\hspace{.2in}" or \verb"\hspace{1.5pc}". To get the caption at the side of the figure add the small horizontal space after the \verb"\includegraphics" command and then put the \verb"\caption" within a minipage of the appropriate width aligned bottom, i.e. \verb"\begin{minipage}[b]{3in}" etc (see code in this file used to generate figures 1--3).

Note that it may be necessary to adjust the size of the figures (using optional arguments to \verb"\includegraphics", for instance \verb"[width=3in]") to get you article to fit within your page allowance or to obtain good page breaks.

\begin{figure}[h]
\begin{minipage}{14pc}
\includegraphics[width=14pc]{name.eps}
\caption{\label{label}Figure caption for first of two sided figures.}
\end{minipage}\hspace{2pc}%
\begin{minipage}{14pc}
\includegraphics[width=14pc]{name.eps}
\caption{\label{label}Figure caption for second of two sided figures.}
\end{minipage} 
\end{figure}

\begin{figure}[h]
\includegraphics[width=14pc]{name.eps}\hspace{2pc}%
\begin{minipage}[b]{14pc}\caption{\label{label}Figure caption for a narrow figure where the caption is put at the side of the figure.}
\end{minipage}
\end{figure}

Using the graphicx package figures can be included using code such as:
\begin{verbatim}
\begin{figure}
\begin{center}
\includegraphics{file.eps}
\end{center}
\caption{\label{label}Figure caption}
\end{figure}
\end{verbatim}

\section*{References}
\begin{thebibliography}{9}
\bibitem{iopartnum} IOP Publishing is to grateful Mark A Caprio, Center for Theoretical Physics, Yale University, for permission to include the {\tt iopart-num} \BibTeX package (version 2.0, December 21, 2006) with  this documentation. Updates and new releases of {\tt iopart-num} can be found on \verb"www.ctan.org" (CTAN). 
\end{thebibliography}

\end{document}


