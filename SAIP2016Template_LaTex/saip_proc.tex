\documentclass[a4paper]{jpconf}
\usepackage{graphicx}
\begin{document}
\title{Search for $tWZ$ production in the Full Run 2 ATLAS dataset using events with four leptons}

\author{Jake Reich}

\address{
Supervisor: Dr. James Keaveney\\
Co-Supervisor: Dr. Sahal Yacoob \\ \vspace{0.2cm}
University of Cape Town (UCT)}

\ead{jake.reich@cern.ch}

\begin{abstract}
All articles {\it must} contain an abstract. This document describes the  preparation of a conference paper to be published in \jpcs\ using \LaTeXe\ and the \cls\ class file. The abstract text should be formatted using 10 point font and indented 25 mm from the left margin. Leave 10 mm space after the abstract before you begin the main text of your article. The text of your article should start on the same page as the abstract. The abstract follows the addresses and should give readers concise information about the content of the article and indicate the main results obtained and conclusions drawn. As the abstract is not part of the text it should be complete in itself; no table numbers, figure numbers, references or displayed mathematical expressions should be included. It should be suitable for direct inclusion in abstracting services and should not normally exceed 200 words. The abstract should generally be restricted to a single paragraph. Since contemporary information-retrieval systems rely heavily on the content of titles and abstracts to identify relevant articles in literature searches, great care should be taken in constructing both.
\end{abstract}

\section{Introduction}
These guidelines show how to prepare articles for publication in \jpcs\ using \LaTeX\ so they can be published quickly and accurately. Articles will be refereed by the \corg s but the accepted PDF will be published with no editing, proofreading or changes to layout. It is, therefore, the author's responsibility to ensure that the content and layout are correct.  This document has been prepared using \cls\ so serves as a sample document. The class file and accompanying documentation are available from \verb"http://jpcs.iop.org".

\section{Preparing your paper}
\verb"jpconf" requires \LaTeXe\ and  can be used with other package files such
as those loading the AMS extension fonts 
\verb"msam" and \verb"msbm" (these fonts provide the 
blackboard bold alphabet and various extra maths symbols as well as 
symbols useful in figure captions); an extra style file \verb"iopams.sty" is 
provided to load these packages and provide extra definitions for bold Greek letters. 

\end{document}