
\section{Data and Monte Carlo Simulation}
\subsection{Data Samples}
Brief overview of Full Run 2 and where the data comes from and what time period.\\\\
Luminosity of full run 2. 

\subsection{Monte Carlo Samples}
Signal and Background samples. Choice of cuts at ntuple level. Specifically why each background is chosen (and why others excluded) $\rightarrow$ e.g. talk about branching fractions, cross sections, topology. How are these backgrounds passing our event selection (e.g. ttz $\rightarrow$ b can be lost/untagged/mis-id'ed) $\rightarrow$ provide an explanation for each background.\\\\
Details of each sample (event generator, parton shower).

\section{Objects}
In the subsections below:\\\\
Explain why we applied each cut/selection.\\\\


\subsection{Leptons}
Tight/loose/med definitions, efficiency of electron and muons specifically at ATLAS. Why we don't consider taus.
\subsection{Jets}
What algorithm did we use and why
\subsection{b-tagging}
What algorithm/WP did we use and why
\section{Kinematic Pre-selection cuts}
Mass windows on Z (OSSF), sum charge $=$ 0, explanations on all other non object level cuts/selections, OSSF $<$ 10 GeV cut
\section{Regions and Event Selection}

Summary table of event selection. Why chose ZZb and ttz region. 
\section{Machine Learning Techniques}
What tool did we use, how did we use it, parameters of bdt/nn, input variables/importance, conversion of event level bdt output (bdtscore) to variable for fitting. Used ROC curve integral as a proxy for how good bdt was doing. 
\section{Fake Lepton Estimation} 
Expected to be a small effect, why? Brief, general explanation/idea of methods used (full explanation/description of what we did and the results/plots/etc. later) 
\section{Analysis Framework: TRExFitter (TRF)}
What is TRExFitter? What can it do? At which stage(s) in the analysis did we use it? Which version did we use? Binning method. Explain calculation of error bars in TRF. \\\\ 
Include general flow chart of analysis (not sure where)








