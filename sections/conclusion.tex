 
%Summary of study. Possible ways to improve/extend analysis. Better understand ttz background? Is there anything we can say about Z, t, b, W, whatever that we can take from the study (e.g. x is difficult to detect/discriminate in such analyses due to a and b, however it can be improved by doing y). Maybe talk about what preliminary studies showed, but couldn't fully implement that idea due to whatever restriction/limitation.

The $tWZ$ process in an important process in the search for new physics since its cross section is sensitive to the charged and neutral couplings to the top quark, which is strongly coupled to the Higgs boson. The top quark's couplings are often modified in many scenarios of new physics that aim to resolve the Hierarchy Problem, therefore a constraint placed on the cross section of $tWZ$ is expected to be impactful in constraining such BSM models. A search for $tWZ$ production using 139 fb$^{-1}$ of $pp$ collision data at a centre-of-mass energy of $\sqrt{s} = 13$ TeV, recorded by the ATLAS experiment at CERN, has been presented. This thesis targeted the tetralepton final state channel. Two SRs and three CRs were defined. Two SRs, instead of one, were defined in order to suppress and constrain the $ZZ$ background. The dominant background processes, $t\bar{t}Z $ and $ZZ$ were constrained by the definition of $t\bar{t}Z$ and $ZZ$ CRs, respectively. The dominant source of fake leptons, originating from the $t\bar{t}Z$ background, was constrained by the $(tWZ)_{\text{fake}}$ CR, using the MC template method. Two BDTs were implemented: an object-level BDT which aims to classify between $\ell b$ systems coming from top quarks and an event-level BDT which aims to discriminate between $tWZ$ and our major backgrounds, $t\bar{t}Z$ and $ZZ$. The output from the object-level BDT was converted to an event-level variable to be used as input to the event-level BDT. A kinematic reconstruction algorithm, 2$\nu$SM, was used to reconstruct top quarks in order to discriminate between $tWZ$ and $t\bar{t}Z$. The output from this algorithm was used as an input variable to the event-level BDT. The trained BDT was shown to discriminate well between signal and background events. Using a modified Asimov dataset in the SRs and real data in the CRs, a blinded maximum-likelihood fit was performed across all regions in the tetralepton channel. The best-fit value of the signal strength in the tetralepton channel was,
\begin{equation}
  \mu (tWZ) =   1.91^{+0.95}_{-0.82}
\end{equation}
with an expected significance of $1.44\sigma$. The expected upper limit on the signal strength of $tWZ$ in the tetralepton channel was,

\begin{equation}
  \mu_{up}^{exp} =   1.61^{+2.35}_{-1.16}
\end{equation}
To further increase the sensitivity of $tWZ$, a blinded maximum-likelihood fit was performed across all regions across the trilepton (studied in an independent analysis by Benjamin Warren (UCT)~\cite{ben-thesis}) and tetralepton channels. The best-fit value of the signal strength across both the trilepton and tetralepton channels were,

\begin{equation}
  \mu (tWZ) =   1.80^{+0.70}_{-0.65}
\end{equation}
with an expected significance of $1.61\sigma$. The expected upper limit on the signal strength of $tWZ$ across both the trilepton and tetralepton channels were,

\begin{equation}
  \mu_{up}^{exp} =   1.43^{+2.04}_{-1.03}
\end{equation}
Although this result does not satisfy the 3$\sigma$ evidence nor the 5$\sigma$ discovery standards, an unblinding of this analysis in the future may still provide the tightest ever constraint on the $tWZ$ process. The results in this analysis are heavily statistically limited, it is therefore expected that future analyses of this process, using larger datasets (such as that from the HL-LHC), would significantly improve the results.










