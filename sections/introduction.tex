 
%Write similar to what is in nrf application. \\\\
%Talk about previous paper's (\tWZ 3-lep) findings - http://cds.cern.ch/record/2625170\\\\
%Explain that SM aims to describe fundamental physics, but fails in certain cases (DM, gravity etc. )\\\\
%Possibly talk about EFT? Finding \tWZ cross section -> global fit.
%FOR REFERENCE:\\
%The production of a single top quark in association with a W± and Z boson (tW±Z) is sensitive to both the neutral and charged electroweak couplings of the top quark as the process involves the simultaneous production of a W boson and a Z boson in association with the top quark. Due to the very large coupling of the top quark to the Higgs boson, the electroweak couplings of the top quark are a theoretically well-motivated area to expect the first signs of new physics. The recent lack of signs of new physics from LHC data tells us that new physics is either very heavy, or is very weakly coupled to Standard Model particles, therefore we might only observe signs of new physics in anomalous rates of well-chosen processes. A prime example of such a process is tWZ. This has an extremely low production cross section (0.7 fb), meaning that it is an extremely rare process to observe and subsequently, it has never been observed by any particle physics experiment. However, the latest datasets recorded by ATLAS are sufficiently large to allow a potential observation of this rare, novel process. 

%We aim to use the Full Run 2 dataset recorded by the ATLAS experiment at CERN to search for the production of a top quark together with a W± and Z boson for the channel with four leptons (two originating from the decay of the Z boson, one from the associated W boson and one from the W boson which decays from the top quark (together with a b quark)). The Standard Model of particle physics has been confirmed to an extraordinary degree of precision, however we know there are stark deficiencies therein. These include its incompatibility with the theory of gravity and an explanation of the matter-antimatter asymmetry in the universe. Especially relevant is the Standard Model's lack of an explanation for the vast differences in the strengths of the fundamental forces (The Hierarchy Problem), constraining the electroweak couplings of the top quark squarely addresses this fundamental scientific question.