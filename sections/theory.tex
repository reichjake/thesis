\section{Standard Model of Particle Physics}
What is SM (renormalisable qft), come from symmetry, brief description of group structure? Explain structure/properties (fermions, bosons, etc.), coupling constants. 'Particles carry color/electric charge, some particles can interact with others/themselves, some can't' Where does it how? How well does it work? Where doesn't it work? 
\subsection{Electroweak Theory}
Properties of W and Z. Decay channels. Z as the standard candle (distinct OSSF lepton signal).
\subsection{Top Quark}
Properties. History (when/how was it discovered/theorised). Why interesting (large mass). Hierarchy problem. Decay channels. Extremely short lifetime (makes b's so important for top ID).
\section{$tWZ$}
\subsection{Tetra-lepton Channel}
Provide feynman diagram. Cross section. 

\subsection{Comparison to Tri-lepton Channel}

Less backgrounds to deal with (in tetralepton). However lower stats (in tetra). Give cross sections (and feynman diagram). Maybe talk a bit about analysis related challenges (trilepton has a hadronically decaying W, does this make the analysis easier or more difficult?).

\section{Effective Field Theory (EFT)}
Brief overview of EFT. What is EFT? Why important to pp as a whole? Why important in twz (high sensitivity to wilson coefficients, expected to have a large impact on global fit)? Similar to what james says in INT note.
\section{Machine Learning in the Context of Particle Physics Analyses}
Brief overview of ML as a whole; History; increase in popularity in recent years (why increased in popularity $\rightarrow$ novel techniques developed, increase in computing power for your buck). Where does it fit into pp (event selection, object reconstruction and ID (jet reco, b-tagging)). Explain concepts (vocabulary), overtraining, training, testing, classifier, classification. Why use x train/test ratio in pp (use some for analysis/use some for training)? Popular tools which are used today (scikit learn, TMVA, xgboost what is theano, what is keras).\\\\
Maybe a subsection on bdt (if end up using it) on the specific algorithm, and minimizing cost function (general). Explanation on ROC curve and why we can use it as a proxy to determine how well our bdt/nn is doing (and where it fails/can fail $\rightarrow$ things to be aware of cautious of when straight up using ROC integral (e.g. overtraining )).
\section{Statistical Techniques}
Brief overview: frequentist and bayes approach in general in pp, why we use frequentist in this analysis

\subsection{Maximum Likelihood Fitting}
Go through theory. Varying histograms.
\subsection{p-value and $\chi^2$}
Go through theory. Go through story of getting p-value, what it means, getting chi-squared, what it means, where the chi squared distribution comes in, degrees of freedom, what is a 'good' chi squared value and what is 'bad' and what do different values mean/infer/suggest.
\subsection{Significance}
Go through theory. What is means/interpreted as. 3 sigma observation, 5 sigma discovery. 
\subsection{Limit Setting} 
Go through theory. 







