\section{Standard Model of Particle Physics}
What is SM (renormalisable qft), come from symmetry, brief description of group structure? Explain structure/properties (fermions, bosons, etc.), coupling constants. 'Particles carry color/electric charge, some particles can interact with others/themselves, some can't' Where does it how? How well does it work? Where doesn't it work? 
\subsection{Electroweak Theory}
Properties of W and Z. Decay channels. Z as the standard candle (distinct OSSF lepton signal).
\subsection{Top Quark}
Properties. History (when/how was it discovered/theorised). Why interesting (large mass). Hierarchy problem. Decay channels. Extremely short lifetime (makes b's so important for top ID).
\section{$tWZ$}
\subsection{Tetra-lepton Channel}
Provide feynman diagram. Cross section. 

The leading order Feynman diagram for \tWZ in the tetra-lepton channel is shown below.




\begin{tikzpicture}
\label{fig:twz_FeymanDiagram}
\begin{feynman}
\vertex (a);
\vertex [below right=of a] (c);
\vertex [below left=of c] (b);

\vertex [right=of c] (d);  %d: tW vertex

\vertex (e) at (4.6,1);  %e: top vertex

\vertex (i) at (7,-1.25);  %i: Z -> ll vertex
\vertex [above right=of i] (f1) {$\ell^{\mp}$}; 
\vertex [below right=of i] (f2) {$\ell^{\pm}$}; 


\vertex (f) at (6,2.35);  %f: Wb vertex
\vertex [below right=of f] (f7) {$b$};

\vertex [above right=of f] (g);  %g: l \nu vertex
\vertex [below right=of g] (f3) {${\nu}_{\ell}$};
\vertex [above right=of g] (f4) {$\ell^{+}$};

%\vertex [below right=of d] (h); %h: l \nu vertex (bottom)
\vertex (h) at (7,-5.1); %h: l \nu vertex (bottom)
\vertex [below right=of h] (f5) {$\bar{\nu}_{\ell}$};
\vertex [above right=of h] (f6) {$\ell^{-}$};



\diagram* {
	(a) -- [fermion, edge label=$b$] (c),
	(b) -- [gluon, edge label'=$g$] (c),
	(c) -- [fermion, edge label=$b$] (d),
	(d) -- [fermion, edge label=$t$] (e),
	
	(e) -- [boson, edge label=$Z$] (i),
	(i) -- [fermion] (f1),
	(i) -- [anti fermion] (f2),
	
	(f) -- [boson, edge label=$W^{+}$] (g),
	(e) -- [fermion] (f),
	(f) -- [fermion] (f7),
	(g) -- [fermion] (f3),
	(g) -- [anti fermion] (f4),
	
	
	
	(d) -- [boson, edge label=$W^{-}$] (h),
 	(h) -- [anti fermion] (f5),
	(h) -- [fermion] (f6),
	
	
	
	
};
\end{feynman}

\end{tikzpicture}

\subsubsection{Backgrounds}

\begin{tikzpicture}
\label{fig:ttz_FeymanDiagram}

\begin{feynman}

\vertex (a);
\vertex [below right=of a] (c);
\vertex [below left=of c] (b);

\vertex [right=of c] (d);   %g -> ttbar vertex

\vertex (e) at (7,3);   %d: t-> Wb vertex
\vertex (f) at (5,-3.5);   %e: t bar-> Z tbar vertex

 %t-> Wb vertex
\vertex [above right=of e] (g);  %(g): W-> l\nu (top)
\vertex [above right=of g] (f1)  {$\ell^{+}$};  
\vertex [below right=of g] (f2)  {$\nu_{\ell}$}; 
\vertex [below right=of e] (f3)  {$b$};   

\vertex (h) at (8,-0.7);   %(h): Z-> ll vertex 
\vertex [above right=of h] (f4) {$\ell^{\mp}$}; 
\vertex [below right=of h] (f5)  {$\ell^{\pm}$}; 


\vertex (i) at (7,-5.5);   %(i): tbar-> bW
\vertex [above right=of i] (f6)  {$\bar{b}$};  
\vertex [below right=of i] (j);   %W-> l\nu vertex (tbar)
\vertex [above right=of j] (f7)  {$\ell^{-}$};  
\vertex [below right=of j] (f8)  {$\bar{\nu}_{\ell}$}; 





\diagram* {

(a) -- [gluon, edge label=$g$] (c),
(b) -- [gluon, edge label=$g$] (c),

(c) -- [gluon, edge label=$g$] (d),

(d) -- [fermion, edge label =$t$](e),
(d) -- [anti fermion, edge label'=$\bar{t}$](f),

(e) -- [boson, edge label=$W^{+}$] (g),
(e) -- [fermion] (f3),
(g) -- [anti fermion] (f1),
(g) -- [fermion] (f2),

(f) -- [boson, edge label=$Z$] (h),
(h) -- [fermion] (f4),
(h) -- [fermion] (f5),

(f) -- [anti fermion] (i),
(i) -- [anti fermion] (f6),
(i) -- [boson, edge label=$W^{-}$] (j),
(j) -- [fermion] (f7),
(j) -- [anti fermion] (f8),

	
	
	
	
};
\end{feynman}
\end{tikzpicture}

\subsection{Comparison to Tri-lepton Channel}

Less backgrounds to deal with (in tetralepton). However lower stats (in tetra). Give cross sections (and feynman diagram). Maybe talk a bit about analysis related challenges (trilepton has a hadronically decaying W, does this make the analysis easier or more difficult?).

\section{Effective Field Theory (EFT)}
Brief overview of EFT. What is EFT? Why important to pp as a whole? Why important in twz (high sensitivity to wilson coefficients, expected to have a large impact on global fit)? Similar to what james says in INT note.
\section{Machine Learning in the Context of Particle Physics Analyses}
Brief overview of ML as a whole; History; increase in popularity in recent years (why increased in popularity $\rightarrow$ novel techniques developed, increase in computing power for your buck). Where does it fit into pp (event selection, object reconstruction and ID (jet reco, b-tagging)). Explain concepts (vocabulary), overtraining, training, testing, classifier, classification. Why use x train/test ratio in pp (use some for analysis/use some for training)? Popular tools which are used today (scikit learn, TMVA, xgboost what is theano, what is keras).\\\\
Maybe a subsection on bdt (if end up using it) on the specific algorithm, and minimizing cost function (general). Explanation on ROC curve and why we can use it as a proxy to determine how well our bdt/nn is doing (and where it fails/can fail $\rightarrow$ things to be aware of cautious of when straight up using ROC integral (e.g. overtraining )).
\section{Statistical Techniques}
Brief overview: frequentist and bayes approach in general in pp, why we use frequentist in this analysis

\subsection{Maximum Likelihood Fitting}
Go through theory. Varying histograms.
\subsection{p-value and $\chi^2$}
Go through theory. Go through story of getting p-value, what it means, getting chi-squared, what it means, where the chi squared distribution comes in, degrees of freedom, what is a 'good' chi squared value and what is 'bad' and what do different values mean/infer/suggest.
\subsection{Significance}
Go through theory. What is means/interpreted as. 3 sigma observation, 5 sigma discovery. 
\subsection{Limit Setting} 
Go through theory. 







