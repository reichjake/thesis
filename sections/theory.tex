\section{Standard Model of Particle Physics}
What is SM (renormalisable qft), come from symmetry, brief description of group structure? Explain structure/properties (fermions, bosons, etc.), coupling constants. 'Particles carry colour/electric charge, some particles can interact with others/themselves, some can't' Where does it how? How well does it work? Where doesn't it work? 
\subsection{Electroweak Theory}
Properties of W and Z. Decay channels. Z as the standard candle (distinct OSSF lepton signal).
\subsection{Top Quark}
Properties. History (when/how was it discovered/theorised). Why interesting (large mass). Hierarchy problem. Decay channels. Extremely short lifetime (makes b's so important for top ID).
\section{$tWZ$}
\subsection{Tetra-lepton Channel}
The leading order Feynman diagram for \tWZ in the tetra-lepton channel is shown below.

\begin{figure}[h!]
	
\centering
\begin{tikzpicture}[thick,scale=0.6, every node/.style={transform shape}]
\label{fig:twz4lep_FeymanDiagram}
\begin{feynman}
\vertex (a) {$b$};
\vertex [below right=of a] (c);
\vertex [below left=of c] (b) {$g$};

\vertex [right=of c] (d);  %d: tW vertex

\vertex [above right=3cm of d] (e);  %e: top vertex

\vertex [below right=3cm of e] (i);  %i: Z -> ll vertex
\vertex [above right=of i] (f1) {$\ell^{\mp}$}; 
\vertex [below right=of i] (f2) {$\ell^{\pm}$}; 


\vertex [above right=2cm of e] (f);  %f: Wb vertex
\vertex [below right=of f] (f7) {$b$};

\vertex [above right=of f] (g);  %g: l \nu vertex
\vertex [below right=of g] (f3) {${\nu}_{\ell}$};
\vertex [above right=of g] (f4) {$\ell^{+}$};


\vertex  [below right=6cm of d] (h); %h: l \nu vertex (bottom)
\vertex [below right=of h] (f5) {$\bar{\nu}_{\ell}$};
\vertex [above right=of h] (f6) {$\ell^{-}$};



\diagram* {
	(a) -- [fermion] (c),
	(b) -- [gluon] (c),
	(c) -- [fermion, edge label=$b$] (d),
	(d) -- [fermion, edge label=$t$] (e),
	
	(e) -- [boson, edge label=$Z$] (i),
	(i) -- [fermion] (f1),
	(i) -- [anti fermion] (f2),
	
	(f) -- [boson, edge label=$W^{+}$] (g),
	(e) -- [fermion] (f),
	(f) -- [fermion] (f7),
	(g) -- [fermion] (f3),
	(g) -- [anti fermion] (f4),
	
	
	
	(d) -- [boson, edge label=$W^{-}$] (h),
 	(h) -- [anti fermion] (f5),
	(h) -- [fermion] (f6),
	
	
	
	
};
\end{feynman}

\end{tikzpicture}
\caption{Example Feynman diagram of \tWZ production in the tetra-lepton channel.}
\end{figure}



\subsubsection{Backgrounds}
The main backgrounds for \tWZ (tetra-lepton channel) are the production of a two tops, both in the $\ell \nu b$ final state channel, together with a $Z$ boson (\ttZ) and diboson production with fully leptonic final states (\ZZ).

\begin{minipage}{.5\textwidth}
	\centering
\begin{tikzpicture}[thick,scale=0.6, every node/.style={transform shape}]
\label{fig:ttz_FeymanDiagram}

\begin{feynman}

\vertex (a) {$g$};
\vertex [below right=of a] (c);
\vertex [below left=of c] (b) {$g$};

\vertex [right=of c] (d);   %g -> ttbar vertex

\vertex [above right=3cm of d] (e);   %d: t-> Wb vertex
\vertex [below right=3cm of d] (f);   %e: t bar-> Z tbar vertex

%t-> Wb vertex
\vertex [above right=3cm of e] (g);  %(g): W-> l\nu (top)
\vertex [above right=of g] (f1)  {$\ell^{+}$};  
\vertex [below right=of g] (f2)  {$\nu_{\ell}$}; 
\vertex [below right=of e] (f3)  {$b$};   

\vertex [above right=3cm of f] (h);   %(h): Z-> ll vertex 
\vertex [above right=of h] (f4) {$\ell^{\mp}$}; 
\vertex [below right=of h] (f5)  {$\ell^{\pm}$}; 


\vertex [below right=2cm of f] (i);   %(i): tbar-> bW
\vertex [above right=of i] (f6)  {$\bar{b}$};  
\vertex [below right=2cm of i] (j);   %W-> l\nu vertex (tbar)
\vertex [above right=of j] (f7)  {$\ell^{-}$};  
\vertex [below right=of j] (f8)  {$\bar{\nu}_{\ell}$}; 





\diagram* {
	
	(a) -- [gluon] (c),
	(b) -- [gluon] (c),
	
	(c) -- [gluon, edge label=$g$] (d),
	
	(d) -- [fermion, edge label =$t$](e),
	(d) -- [anti fermion, edge label'=$\bar{t}$](f),
	
	(e) -- [boson, edge label=$W^{+}$] (g),
	(e) -- [fermion] (f3),
	(g) -- [anti fermion] (f1),
	(g) -- [fermion] (f2),
	
	(f) -- [boson, edge label=$Z$] (h),
	(h) -- [fermion] (f4),
	(h) -- [fermion] (f5),
	
	(f) -- [anti fermion] (i),
	(i) -- [anti fermion] (f6),
	(i) -- [boson, edge label=$W^{-}$] (j),
	(j) -- [fermion] (f7),
	(j) -- [anti fermion] (f8),
	
	
	
	
	
};
\end{feynman}
\end{tikzpicture}
\end{minipage}% This must go next to `\end{minipage}`
\begin{minipage}{.5\textwidth}
	\centering
	\begin{tikzpicture}[thick,scale=0.6, every node/.style={transform shape}]
	\label{fig:zz_FeymanDiagram}
	
	\begin{feynman}
	
	\vertex (a) {$q$};
	\vertex [below right=of a] (c);
	\vertex [below left=of c] (b) {$\bar{q}$};
	
	\vertex [above right=3cm of d] (e); % Z vertex (top)
	\vertex [above right=of e] (f1) {$\ell^{\mp}$}; 
	\vertex [below right=of e] (f2) {$\ell^{\pm}$}; 
	
	
	\vertex [below right=3cm of d] (f); % Z vertex (bottom)
	\vertex [above right=of f] (f3) {$\ell^{\mp}$}; 
	\vertex [below right=of f] (f4) {$\ell^{\pm}$}; 
	
	
	
	
	\diagram* {
		
		(a) -- [fermion] (c),
		(b) -- [anti fermion] (c),
		
		(c) -- [boson, edge label=$Z/ \gamma$] (d),
		
		(d) -- [boson, edge label=$Z$] (e),
		(d) -- [boson, edge label=$Z$] (f),
		
		(e) -- [fermion] (f1),
		(e) -- [anti fermion] (f2),
		
		(f) -- [fermion] (f3),
		(f) -- [anti fermion] (f4)
		
			
	};
	\end{feynman}
	\end{tikzpicture}

\end{minipage}

\captionof{figure}{Example Feynman diagrams for \ttZ (left) and \ZZ (right) in the tetra-lepton channel.}  %minipage 'figure' caption

\ttZ contains fours leptons and two $b$-quarks in its final state (inclusive $\sigma(t\bar{t}Z)= 0.95 \pm 0.08_{\text{stat}} \pm 0.10_{syst}$pb at $\sqrt{s}=13$TeV~\cite{ttz-xsec-paper}) and can easily mimic the \tWZ signal process, for instance, by one of its $b$-jets getting missed during detection. \ZZ contains four leptons and zero $b$-quarks in its final state (inclusive $\sigma(ZZ)= 14.6^{+1.9}_{-1.8}(\text{stat})^{+0.5}_{-0.3}(\text{syst})\pm 0.2 (\text{theo})\pm 0.4 (\text{lumi})$pb at $\sqrt{s}=13$TeV ~\cite{zz-xsec-paper}). One way in which \ZZ can mimic the \tWZ signal process is by reconstruction of a non-prompt $b$-jet. 



\subsection{Comparison to Tri-lepton Channel}

Less backgrounds to deal with (in tetralepton). However lower stats (in tetra). Give cross sections (and feynman diagram). Maybe talk a bit about analysis related challenges (trilepton has a hadronically decaying W, does this make the analysis easier or more difficult?).





\begin{figure}[h!]
	
	\centering
	\begin{tikzpicture}[thick,scale=0.6, every node/.style={transform shape}]
	\label{fig:twz3lep_FeymanDiagram}
	\begin{feynman}
	\vertex (a) {$b$};
	\vertex [below right=of a] (c);
	\vertex [below left=of c] (b) {$g$};
	
	\vertex [right=of c] (d);  %d: tW vertex
	
	\vertex [above right=3cm of d] (e);  %e: top vertex
	
	\vertex [below right=3cm of e] (i);  %i: Z -> ll vertex
	\vertex [above right=of i] (f1) {$\ell^{\mp}$}; 
	\vertex [below right=of i] (f2) {$\ell^{\pm}$}; 
	
	
	\vertex [above right=2cm of e] (f);  %f: Wb vertex
	\vertex [below right=of f] (f7) {$b$};
	
	\vertex [above right=of f] (g);  %g: l \nu vertex
	\vertex [below right=of g] (f3) {${\nu}_{\ell}, q$};
	\vertex [above right=of g] (f4) {$\ell^{+}, \bar{q}$};
	
	
	\vertex  [below right=6cm of d] (h); %h: l \nu vertex (bottom)
	\vertex [below right=of h] (f5) {$\bar{q}, \bar{\nu}_{\ell}$};
	\vertex [above right=of h] (f6) {$q, \ell^{-}$};
	
	
	
	\diagram* {
		(a) -- [fermion] (c),
		(b) -- [gluon] (c),
		(c) -- [fermion, edge label=$b$] (d),
		(d) -- [fermion, edge label=$t$] (e),
		
		(e) -- [boson, edge label=$Z$] (i),
		(i) -- [fermion] (f1),
		(i) -- [anti fermion] (f2),
		
		(f) -- [boson, edge label=$W^{+}$] (g),
		(e) -- [fermion] (f),
		(f) -- [fermion] (f7),
		(g) -- [fermion] (f3),
		(g) -- [anti fermion] (f4),
		
		
		
		(d) -- [boson, edge label=$W^{-}$] (h),
		(h) -- [anti fermion] (f5),
		(h) -- [fermion] (f6),
		
		
		
		
	};
	\end{feynman}
	
	\end{tikzpicture}
	\caption{Example Feynman diagram of \tWZ production in the tri-lepton channel.}
\end{figure}

The most apparent difference between the tri and tetra-lepton channels is the amount of statistics present, with the tetra-lepton channel having far less events in its phase space than that of the tri-lepton channel. The lack of statistics in the tetra-lepton channel can be attributed to its low production cross section, $\sigma^{\text{NLO}}_{(tW^{\pm}Z).Br(4\ell)} = \SI{0.7}{fb}$\cite{twz_3_lep}. The tri-lepton channel has a production cross section ($\sigma^{\text{NLO}}_{(tW^{\pm}Z).Br(3\ell)} = \SI{3.9}{fb}$\cite{twz_3_lep}) around a factor of 4 larger than that of the tetra-lepton channel. This difference between the production cross section of the two decay channels can be largely attributed to the difference in branching ratios ($\frac{\Gamma_i}{\Gamma}$) between a hadronically decaying $W$ boson, $\frac{\Gamma_{W \rightarrow had}}{\Gamma_W} = (67.41 \pm 0.27) \%$\cite{pdg}, present in the tri-lepton channel and a leptonically decaying $W$ boson, $\frac{\Gamma_{W \rightarrow \ell^{\+} \nu}}{\Gamma_W}  = (10.86 \pm 0.09) \%$\cite{pdg}, present in the tetra-lepton channel.\\\\
Despite the tetralepton channel's low statistics, it is not subject to the large $WZ$ background present in the trilepton channel. The tetralepton channel has a relatively large $ZZ$ background (not present in the trilepton channel), fortunately this can be easily suppressed due to the full reconstructability of the two leptonically decaying $Z$-bosons.


\section{Effective Field Theory (EFT)}
Brief overview of EFT. What is EFT? Why important to pp as a whole? Why important in twz (high sensitivity to wilson coefficients, expected to have a large impact on global fit)? Similar to what james says in INT note.
\section{Machine Learning in the Context of Particle Physics Analyses}
Brief overview of ML as a whole; History; increase in popularity in recent years (why increased in popularity $\rightarrow$ novel techniques developed, increase in computing power for your buck). Where does it fit into pp (event selection, object reconstruction and ID (jet reco, b-tagging)). Explain concepts (vocabulary), overtraining, training, testing, classifier, classification. Why use x train/test ratio in pp (use some for analysis/use some for training)? Popular tools which are used today (scikit learn, TMVA, xgboost what is theano, what is keras).\\\\
Maybe a subsection on bdt (if end up using it) on the specific algorithm, and minimizing cost function (general). Explanation on ROC curve and why we can use it as a proxy to determine how well our bdt/nn is doing (and where it fails/can fail $\rightarrow$ things to be aware of cautious of when straight up using ROC integral (e.g. overtraining )).
\section{Statistical Techniques}
Brief overview: frequentist and bayes approach in general in pp, why we use frequentist in this analysis

\subsection{Maximum Likelihood Fitting}
Go through theory. 
\subsection{p-value and $\chi^2$}
Go through theory. Go through story of getting p-value, what it means, getting chi-squared, what it means, where the chi squared distribution comes in, degrees of freedom, what is a 'good' chi squared value and what is 'bad' and what do different values mean/infer/suggest.
\subsection{Significance}
Go through theory. What is means/interpreted as. 3 sigma evidence, 5 sigma discovery. 
\subsection{Limit Setting} 
Go through theory. 







